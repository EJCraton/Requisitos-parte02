\chapter{Conclusão}

  A adoção do processo de Engenharia de requisitos utilizando uma metodologia ágil tornou se claro a import\^{a}ncia da utilização rigorosa de
  todas as etapas desse processo de gerenciamento dos requisitos para garantir uma melhor qualidade do produto de software.

\section{Experiência de execução das técnicas de elicitação de requisitos}

  Os conhecimento adquiridos sobre as técnicas de elicitação, que é a parte inicial e talvez a mais essencial do início de um sistema de
  software, foram de fundamental import\^{a}ncia para a execução do projeto, entre eles, as entrevistas, que por ser uma técnica tradicional e
  já conhecida e praticada por todos, foi fundamental para entender o contexto, as necessidades do cliente e o problema, tudo isso de forma
  direta, o brainstormings que também ajudou bastante a equipa elaborar ideias para o produto, tendo como principal objetivo, gerar ideias sem
  julgamento inicial, e a prototipação que infelizmente não tivemos tempo de faze-lo porém nas próprias entrevistas já deixamos claro essa
  possibilidade de o protótipo ou produto final não ser entregue por completo devido ao tempo de desenvolvimento ser curto e que pode existir a
  possibilidade da equipe continuar o projeto após a execução da disciplina.

\section{Experiência de Execução do Trabalho}

  A disciplina nos proporcionou uma visão de como é a experiência de ter um cliente real e de como descobrir e analisar quais são as reais
  necessidades do cliente, e como gerenciá-las de uma maneira eficaz e eficiente para que o produto criado supra todas as suas necessidades e o
  satisfaça, também foi notável a real dificuldade de lidar com variáveis não controladas, como características culturais, sociais,
  econômicas, humor, profissionalismo entre outras.


  O entendimento do problema é o primeiro passo para uma boa elicitação dos requisitos, sem saber o real problema não tem como satisfazer as
  necessidades do cliente, e isso foi uma dificuldade pois a equipe nunca teve contato nenhum com a área de geologia e no começo ninguém
  entendia nada do que o cliente queria, pois se tratava de um produto que seria específico para as necessidades daquele cliente, então tivemos
  que estudar um pouco sobre geologia e entender de fato o problema que eles queriam resolver e a empresa nos ajudou bastante a entender o
  contexto na qual eles estavam trabalhando e qual o problema eles tinham em mente.


  No início a equipe estava completamente perdida, pois o conhecimento da engenharia de requisitos não era claro o suficiente para colocar em
  prática, devido a isso tivemos duas reuniões improdutivas na qual focamos o problema errado, porém com o decorrer do semestre os conceitos e
  técnicas foram ficando cada vez mais claro e conseguimos criar e executar o processo, as técnicas de elicitação e conseguimos entender as
  reais necessidades do cliente.

\section{Experiência da Disciplina}

  A elicitação incorreta dos requisitos pode ser de fato o maior problema dentro de um projeto e isso pode custar caro e até mesmo inviabilizar
  o projeto por completo, pois tendo os requisitos incompletos ou que não satisfaçam as reais necessidades do cliente pode fazer com que o
  produto não sirva para nada e além de perder o cliente o prejuízo financeiro será notável.


  A disciplina de engenharia de requisitos foi de fundamental import\^{a}ncia para compreender que essa área além de ser a área mais importante
  dentro de um projeto ela não é tão simples como aparenta ser, a ideia inicial que se tem de coletar requisitos é de que em uma entrevista a
  equipe coleta tudo que o cliente deseja, coloca em andamento o processo e cria um produto para o cliente, porém não é assim que as coisas
  funcionam na prática, pois as vezes o cliente não sabe o que deseja e os requisitos tem que ter um gerenciamento constante de mudança, além de
  que para começar a coletar os requisitos temos que entender perfeitamente o contexto do cliente e o problema na qual ele deseja resolver, com
  isso a equipe consegue ter uma ideia e soluções que agregam valor ao cliente, também a constante comunicação com o cliente e essencial para o
  sucesso do projeto, já que o cliente com o tempo pode mudar suas prioridades e talvez o produto não sirva mais as suas necessidades, causando
  um prejuízo se isso só for descoberto na entrega do produto.

