\chapter{Gerência de requisitos}

  Neste capítulo serão apresentados os requisitos elicitados, desde os épicos, localizados no mais alto nível do SAFe até as
  histórias de usuário no nível Team, o nível mais abaixo no SAFe.

  Para a gerência de requisitos, foi utilizada a ferramenta TargetProcess, como foi proposto no trabalho 1.

\section{Nível de Portifolio}

  O Portfolio é o mais alto nível nível no SAFe, nesta etapa do trabalho o nível de portfólio é evidenciado pelas seguintes atividades:

  \begin{itemize}
    \item Aprender o contexto da empresa
    \item Identificar o problema
    \item Definir tema de investimento
    \item Definir épicos
    \item Validar épicos
    \item Priorizar épicos
    \item Gerenciar épicos
  \end{itemize}

\subsection{Requisitos elicitados}

  Dado que um dos principais problemas relacionados com a empresa é a falta de espaço físico, necessitando sempre de alguma plataforma remota
  para realizar os trabalhos e guardar seus produtos, isso se tornou um tema de investimento.

  \begin{description}
    \item[Tema de investimento:] \
      \begin{itemize}
        \item Armazenamento adequado de produtos
      \end{itemize}
    \item[Épicos] \
      \begin{itemize}
        \item EP-01: Criar banco de dados
        \item EP-02: Criar Servidor
      \end{itemize}
  \end{description}

  A seguir os épicos estão sendo detalhados através do template \textit{lightwaight business case}

\subsubsection{Épico 01: Criar banco de dados}

  \begin{table}[!htb]
    \centering
    \rowcolors{2}{gray!25}{white}
    \begin{tabular}{rp{10cm}}                    \hline
      \rowcolor{gray!50}
      \multicolumn{2}{c}{Criação de Banco de Dados} \\ \hline
      \textbf{Para}                 & A empresa e seus funcionários                                                             \\
      \textbf{Quem}                 & Faz uso dos produtos da empresa                                                           \\
      \textbf{A}                    & GeoBD                                                                                     \\
      \textbf{É uma}                & Ferramenta de controle e armazenamento de documentos                                      \\
      \textbf{Que}                  & Facilita e Centraliza os documentos                                                       \\
      \textbf{Diferente}            & Usar ferramentas que não controlam acesso                                                 \\
      \textbf{Nossa solução}        & Permite controlar o acesso aos documentos                                                 \\
      \rowcolor{gray!50} \hline
      \multicolumn{2}{c}{Escopo}                    \\ \hline
      \textbf{Critérios de sucesso} & Garantir uma base de armazenamento privada e com capacidade de armazenamento média/alta.  \\
      \textbf{No escopo}            & Controlar e facilitar acesso aos documentos produzidos pela empresa                       \\
      \textbf{Fora do escopo}       & Validação dos arquivos inseridos no banco
    \end{tabular}
    \caption{Épico 01}
  \end{table}

\subsubsection{Épico 02: Criar servidor}

  \begin{table}[!htb]
    \centering
    \rowcolors{2}{gray!25}{white}
    \begin{tabular}{rp{10cm}}                 \hline
      \rowcolor{gray!50}
      \multicolumn{2}{c}{Criação de um servidor} \\ \hline
      \textbf{Para}                 & Empresa e seus funcionários                                                                   \\
      \textbf{Quem}                 & Faz uso dos produtos da empresas                                                              \\
      \textbf{A}                    & GeoBD                                                                                         \\
      \textbf{É uma}                & Plataforma remota                                                                             \\
      \textbf{Que}                  & Gerencia o banco de dados                                                                     \\
      \textbf{Diferente}            & De utilizar o banco de forma local, em somente um computador.                                 \\
      \textbf{Nossa solução}        & Uma plataforma remota que possibilita a modificação do banco de dados                         \\
      \rowcolor{gray!50} \hline
      \multicolumn{2}{c}{Escopo} \\ \hline
      \textbf{Critérios de sucesso} & Ser capaz de armazenar o banco de dados no servidor, levando em conta a capacidade e o nível
                                      de segurança exigida pela empresa.                                                            \\
      \textbf{No escopo}            & Capacidade de acesso a qualquer produto por qualquer membro da empresa.                       \\
      \textbf{Fora do escopo}       & Validação dos arquivos inseridos no banco.
    \end{tabular}
    \caption{Épico 02}
  \end{table}

\section{Nível de programa}

  O Program é o nível intermediário no SAFe, nesta etapa do trabalho o nível de programa é evidenciado pelas seguintes atividades:

  \begin{itemize}
    \item Definir features e enables
    \item Validar features e enables
    \item Definir visão
    \item Priorizar features e enables
    \item Definir roadmap
    \item Planejar incremento do produto
    \item Gerenciar features
  \end{itemize}

\subsection{Requisitos elicitados}

  \begin{itemize}
    \item \textbf{FEA-01}: Gerenciar imagens de satélite
    \item \textbf{FEA-02}: Gerenciar projetos de ArcGIS
    \item \textbf{FEA-03}: Expor informações dos arquivos
    \item \textbf{FEA-04}: Personalizar o banco
    \item \textbf{FEA-05}: Criptografia dos dados
    \item \textbf{FEA-06}: Integridade do banco de dados
    \item \textbf{FEA-07}: Gerenciar usuários
  \end{itemize}

  \begin{table}[!htb]
    \centering
    \rowcolors{2}{gray!25}{white}
    \begin{tabular}{llp{3cm}p{8cm}} \hline
      \rowcolor{gray!50}
      \textbf{Épico} & \textbf{Feature} & \textbf{Descrição} & \textbf{Benefício}                                                     \\ \hline
      EP-01 & FEA-01 & Gerenciar imagens de satélite  & Permitir o usuário total controle do banco de dados, possibilitando a adição
                                                        e remoção de imagens.                                                         \\
      EP-01 & FEA-02 & Gerenciar projetos de ArcGIS   & Permitir o usuário total controle do banco de dados, possibilitando a adição,
                                                        remoção e alteração de projetos.                                              \\
      EP-01 & FEA-03 & Expor informações dos arquivos & Permitir o usuário visualizar as informações sobre os metadados do arquivo.   \\
      EP-01 & FEA-04 & Personalizar o banco           & Permitir a organização dos artefatos mantidos no banco                        \\
      EP-02 & FEA-05 & Criptografia dos dados         & Permitir com que os arquivos sejam armazenados de forma segura.               \\
      EP-02 & FEA-06 & Integridade do banco de dados  & Permitir com que os arquivos sejam recuperados caso ocorra uma transferência
                                                        inadequada.                                                                   \\
      EP-02 & FEA-07 & Gerenciar usuários             & Permitir com que somente usuários cadastrados pela empresa tenham acesso ao
                                                        banco.
    \end{tabular}
    \caption{Tabela de Features}
  \end{table}


\subsection{Visão}

  A visão do produto se encontrar no capitulo \ref{visao}

\subsection{Roadmap}

  Em construção\ldots

\section{Nível de time}

  O Team é o último nível do SAFe, nesta etapa do trabalho o nível de time é evidenciado pelas seguintes atividades:

  \begin{itemize}
    \item Definir histórias de usuários ou US
    \item Validar US
    \item Planejar sprint
    \item Executar sprint
    \item Gerenciar sprint
    \item Gerenciar US
  \end{itemize}

\subsection{Requisitos elicitados}

  Em construção\ldots

\section{Gerência de mudança}

  Falar sobre o targetProcess, inserir algumas imagens, rastreabilidade e atributos dos requisitos e etc\ldots

  Em construção\ldots

