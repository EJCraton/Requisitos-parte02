\chapter{Visão do projeto}

  \label{visao}

\section{Finalidade}

  Este capítulo apresenta os principais fatores que descreve as necessidades em relação a construção da plataforma de sistema online de
  armazenamento de arquivos da própria empresa junior Cráton através de um banco de dados.

\section{Problema}

  A Cráton é uma empresa júnior recente e formada por estudantes do curso de geologia da Universidade de Brasília que atualmente tem 15 membros.
  A empresa tem o objetivo de prestar serviços geológicos, sendo que os atuais produtos são de geoprocessamento e geologia de prospecção.
  Futuramente a Cráton pretende expandir a variedade de seus produtos abrangendo as áreas de espeleologia, geologia ambiental e geoturismo. Com
  isso a Cráton tem como um dentre vários problemas não ter um sistema online de armazenamento da própria empresa na qual hoje eles utilizam do
  google driver para fazer esse armazenamento e como o google driver tem um espaço de armazenamento limitado e alguns arquivos que a empresa
  precisa guardar são de tamanhos variados chegando até a ocupar GB de memória, o espaço do google driver para esse armazenamento se tornou um
  desafio para a empresa.

\subsection{Abordagem do problema}

\subsubsection{Tema de investimento}

  A partir das reuniões realizadas com a presidente da empresa, o Tema de investimento definido foi o armazenamento adequado de produtos. A
  proposta do banco de dados irá resolver o problema do “pequeno” limite de armazenamento do Google Drive, na qual não irá ter que usar uma
  conta compartilhada entre os membros, evitando o compartilhamento de itens pessoais e evitar ainda mais o constante uso do software Arcgis que
  prover as imagens.

\subsubsection{Diagrama de causa e efeito}

  Também conhecido como Fishbone é uma eficiente maneira de segmentação e compreensão de um problema, fornecendo um mecanismo recursivo para
  dividi-lo em problemas menores até que as causas raízes, ou seja, os problemas-chave que geraram os demais sejam identificados.

  Imagem do fishbone se encontra no anexo \ref{fishbone}

\subsubsection{Formulação do problema}

  \begin{table}[!htb]
    \centering
    \rowcolors{2}{gray!25}{white}
    \begin{tabular}{p{5cm}p{8cm}} \hline
      \rowcolor{gray!50}
      \multicolumn{2}{c}{Formulação} \\ \hline
      \textbf{O problema é}           & A empresa não ter um sistema online de armazenamento na qual hoje eles utilizam do google driver
                                        para fazer esse armazenamento e como o google driver tem um espaço de armazenamento limitado e
                                        alguns arquivos que a empresa precisa guardar são de tamanhos variados chegando até a ocupar GB
                                        de memória, o espaço do google driver para esse armazenamento se tornou um desafio para a empresa   \\
      \textbf{Afeta}                  & A empresa júnior                                                                                    \\
      \textbf{Cujo impacto é}         & Prejuízos em relação a organização de arquivos da empresa, podendo perder-los, ou até mesmo tendo
                                        que apagar alguns arquivos para armazenar outros, assim perdendo informações que podia ser útil no
                                        futuro.                                                                                             \\
      \textbf{Uma boa solução seria}  & Um módulo de software que fosse responsável por armazenar esses arquivos de maneira segura e tendo
                                        um limite muito superior ao que eles tem hoje. Tendo um sistema de login para cada membro, evitando
                                        assim o compartilhamento de dados pessoais.
    \end{tabular}
    \caption{Formulação do problema}
  \end{table}

\subsubsection{Sentença de Posição do produto}

  \begin{table}[!htb]
    \centering
    \rowcolors{2}{gray!25}{white}
    \begin{tabular}{p{5cm}p{8cm}} \hline
      \rowcolor{gray!50}
      \multicolumn{2}{c}{Produto}                                                                                                       \\ \hline
      \textbf{Para}           & Empresa Junior Cráton                                                                                   \\
      \textbf{Que}            & Necessita de um espaço de armazenamento grande o suficiente para inserir todos os seus arquivos.        \\
      \textbf{O}              & GeoBD                                                                                                   \\
      \textbf{É um}           & É um banco de dados                                                                                     \\
      \textbf{Que}            & Armazena arquivos dos diferentes formatos                                                               \\
      \textbf{Diferente de}   & Outras aplicações como Google Driver, Drop Box                                                          \\
      \textbf{Nosso produto}  & Terá segurança dos dados, sistema de login, espaço de armazenamento suficiente para qualquer tipo de dado
    \end{tabular}
    \caption{Sentença de posição do produto}
  \end{table}

\subsection{Escopo}

  O escopo do projeto é de certa forma pequeno, principalmente devido ao tempo para realização do produto e também de a empresa ser jovem, logo
  eles não tem uma ideia muito bem estruturada e definida. Como um dentre vários problemas da Cráton é não ter um sistema online de armazenamento
  da própria empresa, através das negociações chegamos na conclusão de que um Banco de Dados da Cráton seria a melhor solução para resolver um
  dos seus principais problemas, melhorar o gerenciamento de arquivos da empresa.

  Todos os membros da empresa geram diversos arquivos de diversos tamanhos, atualmente o gerenciamento se dá através da utilização do
  repositório do Google Drive, que eventualmente eles atingirão o espaço máximo de armazenamento.

\section{Descrição dos envolvidos e dos usuários}

\subsection{Resumo dos usuários}

  \begin{table}[!htb]
    \centering
    \rowcolors{2}{gray!25}{white}
    \begin{tabular}{p{3,5cm}p{6cm}p{6cm}} \hline
      \rowcolor{gray!50}
        \textbf{Nome}           & \textbf{Descrição} & \textbf{Responsabilidade}                                                          \\
      \hline
        Gerentes de Projetos    & Membros da empresa responsáveis por gerenciar cada
                                  um dos projetos e arquivos.                         & Recebem as solicitações, pedidos de orçamento,
                                                                                        dúvidas sobre os projetos. Acompanham prazos e
                                                                                        avaliam riscos de acordo com as metas e recursos
                                                                                        da empresa                                        \\
        Funcionarios da Cráton  & Membros da empresa que desejam armazenar arquivos
                                  de forma segura e eficiente.                        & Cada membro tem suas respectivas responsabilidades
                                                                                        e geram arquivos diferentes para armazenamento.
    \end{tabular}
    \caption{Usuários}
  \end{table}

\subsection{Resumo dos stakeholders}

  Os envolvidos no projeto são aqueles personagens que serão diretamente afetados e contribuirão diretamente para a tomada de decisões que
  levarão a concepção e a construção do software. Podem não ser considerados os usuários diretos da ferramenta, porém terão papéis bem definidos
  no processo de desenvolvimento do componente.

  A abordagem ágil fornece uma série de papéis e responsabilidades que podem ser adaptadas ao contexto do projeto. Levando em consideração as
  discussões apresentadas durante a escolha da metodologia na fase de modelagem do processo, a Tabela abaixo descreve a lista de envolvidos no
  projeto.

  \begin{table}[!htb]
    \centering
    \rowcolors{2}{gray!25}{white}
    \begin{tabular}{p{3,5cm}p{6cm}p{6cm}} \hline
      \rowcolor{gray!50}
        \textbf{Nome}                    & \textbf{Descrição}                               & \textbf{Responsabilidade}                     \\
      \hline
        Gerente de Portfólio de Programa & Representa a pessoa que tem mais impacto
                                           nas decisões tantos estratégicas quanto
                                           financeiras dentro do framework, e entende
                                           os limites da estratégia de negócio da empresa,
                                           de tecnologia e de fundos.                       & Uma de suas responsabilidades é de
                                                                                              participar das sessões de escolha e
                                                                                              comunicação dos temas de investimento
                                                                                              e da definição e priorização do backlog
                                                                                              de épicos                                     \\
        Gerente de Épicos                & Representa o papel na qual toma-se a
                                           responsabilidade de gerenciar épicos individuais
                                           por todo o processo de portifolio kanban,
                                           desenvolvendo casos de negócios.                 & Trabalha-se com os principais stakeholders
                                                                                              na análise de valor agregado do épico.
                                                                                              Quando o épico é aprovado, o gerente de
                                                                                              épico trabalha com o time de desenvolvimento
                                                                                              e o gerente de produto para estabelecer as
                                                                                              atividades de desenvolvimento, para que as
                                                                                              mesmas atinjam os benefícios de negócios do
                                                                                              épico em específico.                          \\
        Arquiteto da empresa             & Representa a pessoa que sempre visa manter uma
                                           visão geral das tecnologias, soluções da empresa
                                           e iniciativas de desenvolvimento.                & Uma de suas atividades é entender e comunicar
                                                                                              os temas de investimento e outras chaves de
                                                                                              negócios para os arquitetos de sistema e
                                                                                              stakeholders não técnicos, e também
                                                                                              influenciar na decisão de uma modelagem comum
                                                                                              e em boas práticas de codificação.            \\
    \end{tabular}
    \caption{Stakeholders em nível de portifolio}
  \end{table}

  \begin{table}[!htb]
    \centering
    \rowcolors{2}{gray!25}{white}
    \begin{tabular}{p{3,5cm}p{6cm}p{6cm}} \hline
      \rowcolor{gray!50}
        \textbf{Nome}                    & \textbf{Descrição}                               & \textbf{Responsabilidade}                     \\
      \hline
        Gerente de Releases              & Representa a pessoa que planeja a release, e
                                           coordena a implementação de todas as capacidades
                                           e funcionalidades nas diversas iterações dentro
                                           de uma release.                                  & Um de seus papéis é comunicar o status da
                                                                                              release para stakeholders externos a empresa,
                                                                                              e também de prover uma autorização final da
                                                                                              release.                                      \\
        Gerente de Produto               & Junto do Gerente de Soluções, eles formam as
                                           principais autoridades de conteúdo.              & Eles criam a visão do programa, trabalham
                                                                                              com os clientes e também com os Product Owner
                                                                                              para entenderem e comunicarem as necessidades,
                                                                                              participam na validação de soluções propostas,
                                                                                              define os requisitos, gerencia e prioriza o
                                                                                              fluxo de trabalho e também define releases e
                                                                                              Program Increments.                           \\
    \end{tabular}
    \caption{Stakeholders em nível de programa}
  \end{table}

  \begin{table}[!htb]
    \centering
    \rowcolors{2}{gray!25}{white}
    \begin{tabular}{p{3,5cm}p{6cm}p{6cm}} \hline
      \rowcolor{gray!50}
        \textbf{Nome}                    & \textbf{Descrição}                               & \textbf{Responsabilidade}                     \\
      \hline
        Product Owner (P.O)              & Representa a pessoa que tem como responsabilidade
                                           a definição das histórias de usuário e de
                                           priorizar o backlog de time.                     & O P.O desempenha um papel importantíssimo no
                                                                                              quesito qualidade, pois é o único do time que
                                                                                              tem a responsabilidade de aceitar as histórias
                                                                                              como finalizadas. É bastante envolvido na
                                                                                              construção do backlog do programa e na
                                                                                              preparação e refinamento                      \\
        Scrum Master                     & Representa a pessoa que está sempre monitorando
                                           a equipe com o intuito de fazer com que todos
                                           sigam a metodologia, ou seja, o Scrum Master
                                           monitora os integrantes para que todos cumpram
                                           e sigam os princípios da metodologia do
                                           planejamento do PI.                              & Uma das responsabilidades do Scrum Master é
                                                                                              manter a equipe focada nos objetivos certos,
                                                                                              assegurando um fluxo de produtividade o mais
                                                                                              alto possível. Também tem como
                                                                                              responsabilidade facilitar os encontros do
                                                                                              time, tanto no planejamento quanto na
                                                                                              retrospectiva e revisão da sprint.            \\
        Desenvolvedores                  & Representa as pessoas que vão de fato construir
                                           o sistema, produzindo o código fonte e os testes
                                           da aplicação.                                    & Participa de todas as atividades do processo
                                                                                              no nível de time.                             \\
        Equipe                           & Representa a junção dos três citados acima       & Participa de todas as atividades do processo
                                                                                              no nível de time.
    \end{tabular}
    \caption{Stakeholders nível de time}
  \end{table}

\subsection{Principais Necessidades dos Usuários}

  O componente de software deve ser projetado sobre as principais necessidades dos usuários, levantadas utilizando técnicas como Brainstorming,
  diagrama Fishbone e reuniões com o cliente. A seguir estão listadas as principais necessidades dos clientes:

  \begin{itemize}
    \item Falta um espaço de armazenamento de arquivos de forma segura, organizada e eficiente.
    \item Esse espaço de armazenamento não pode haver um compartilhamento de arquivos ou dados pessoais.
    \item Espaço de armazenamento com interface de fácil acesso e com boa usabilidade.
    \item Cada usuário da empresa tem que ter uma conta particular para acesso do banco de dados.
    \item Espaço de armazenamento suficiente para guardar arquivos pesados de imagens de satélites.
  \end{itemize}

\subsection{Alternativas e Concorrência}

  Diversas empresas possuem softwares internos que lidam com o gerenciamento e armazenamento de arquivos. Entretanto, o desenvolvimento dessas
  ferramentas é demasiadamente caro para os padrões de uma Empresa Júnior. Sendo assim, a empresa utiliza o Google Drive para o armazenamento de
  seus arquivos, pois é uma ferramenta que está dentro do orçamentários da empresa pelo fato dela ser de graça. Contudo, essa ferramenta não
  solucionou o problema, tendo que o espaço de armazenamento já está no limite.

\section{Perspectiva do Produto}

  Esta Seção apresenta uma ideia resumida do que será a solução. Para a elaboração de tal foram levadas em consideração principalmente as
  necessidades e os objetivos da empresa, as prioridades apresentadas pelo cliente, os recursos financeiros disponíveis, o tempo para se
  produzir o software e as limitações.

\subsection{Descrição da Solução e Recursos do produto}

  A proposta da solução é para resolver o problema do “pequeno” limite de armazenamento do Google Drive, não precisar ter que usar uma conta
  compartilhada entre os membros, evitando o compartilhamento de itens pessoais e evitar ainda mais o constante uso do software Arcgis que
  prover as imagens.

\subsection{Recursos do produto}
 \begin{table}[!htb]
    \centering
    \rowcolors{2}{gray!25}{white}
    \begin{tabular}{p{3,5cm}p{6cm}p{6cm}} \hline
      \rowcolor{gray!50}
        \textbf{Capacidade}           & \textbf{Descrição}                                                           \\
      \hline
        Gerenciamento de artefatos e produtos    & Permite com que a empresa consiga gerenciar imagens de satélites e projetos do ArcGIS, de acordo com o nível de permissão.                                                              \\
        Gerenciamento de usuários  & Permite com que gerencia os usuários e também suas permissões de acessos.                                                                \\
        Garantia da integridade    & Permite a garantia da integridade dos produtos realizados pela empresa, na qual o software realizará semanalmente, o backup dos arquivos armazenados.
                                                                                \\
        Personalização do banco    & Permite com que os artefatos e produtos sejam organizados de maneira customizada pela empresa.                                       \\
                                                                                                                                              

                                                                                       
    \end{tabular}
    \caption{Capacidade}
  \end{table}


  Logo a solução possuirá algumas características principais:

  \begin{itemize}
    \item Sistema de login de cada membro;
    \item Possibilidade de guardar qualquer tipo de arquivo gerado do software que a empresa utiliza para gerar as imagens e shapes, Arcgis;
    \item Mais segurança de seus produtos que serão salvos no banco;
    \item Armazenamento maior para que eles não tenham receio de guardar imagens de satélites.
  \end{itemize}



\section{Restrições do Projeto}

  \begin{itemize}
    \item Há limitação de recursos computacionais, já que a empresa terá que lidar com o empenho de servidores dedicados em serviços de nuvem
      genéricos como Digital Ocean, já que a mesma não possui servidores em seu ambiente empresarial.
    \item Há também carência de uma equipe para realizar a manutenção e a evolução do software pós produção, visto que a equipe de T.I. é
      reduzida e os recursos financeiros são limitados.
  \end{itemize}

