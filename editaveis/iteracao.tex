\chapter{Planejamento da primeira iteração}

  Este capítulo trata das histórias de usuário que foram implementadas na primeira iteração do produto

\section{Planejamento da sprint 01}

  As histórias que serão implementadas nessa sprint são:

   \begin{itemize}
    \item US01: \textbf{Eu} como usuário, \textbf{desejo} salvar uma imagem de satélite. \textbf{Para que eu possa} acessar a qualquer momento e local posteriormente
    \begin{itemize}
    	\item Para salvar uma imagem o usuário deve estar logado no sistema
		\item O usuário deve acessar o menu em “Enviar”
		\item O usuário deve selecionar uma pasta ou projeto para enviar a imagem 
		\item O usuário seleciona a imagem que deseja enviar ao sistema
		\item A imagem não pode possuir o mesmo nome de outra imagem que esteja na mesma pasta ou projeto
		\item O sistema deve confirmar a inclusão, ou avisar caso haja algum problema
    \end{itemize}
    \item US02: \textbf{Eu} como usuário, \textbf{desejo} acessar uma imagem de satélite do banco. \textbf{Para que eu possa} realizar meu trabalho
    \begin{itemize}
    	\item É necessário estar logado e ter permissões para acessar a imagem. 
    	\item Para acessar imagem é necessário ir para a pasta ou projeto onde a mesma se encontre.
    	\item Do lado do nome da imagem deve existir o botão “Baixar”, que executa a ação e baixa a imagem para a pasta de downloads da máquina que está acessando a aplicação.
    \end{itemize}
    \item US03: \textbf{Eu} como usuário, \textbf{desejo} excluir uma imagem de satélite do banco. \textbf{Para que eu possa} organizar melhor meus arquivos
    \begin{itemize}
    	\item É necessário estar logado e ter permissão para deletar a imagem. 
    	\item Para deletar a  imagem é necessário ir para a pasta ou projeto onde a mesma se encontre.
    	\item Do lado do nome da imagem deve existir o botão “Excluir”, que executa a ação e deleta a imagem.
    \end{itemize}
    \item US12: \textbf{Eu} como usuário, \textbf{desejo} me cadastrar na ferramenta. \textbf{Para que eu possa} utilizar a mesma 
    \begin{itemize}
    	\item Ao entrar na aplicação, deve ser mostrado um quadro com as informações necessárias para realizar o cadastro.
    	\item O usuário deve então entrar com os seguintes dados: nome completo, cpf, matrícula, telefone, email e senha. 
    	\item Ao finalizar de preencher os dados devem ser validados de forma que: nome não pode passar de 150 caracteres, somente letras; cpf e matrícula devem ser validados; matrícula com 10 caracteres, apenas números;  e cada pessoa pode possuir apenas um registro no sistema.
    	\item Caso algum dado digitado não passe na validação uma mensagem deve ser informada ao usuário pelo o motivo do dado não ter sido aceito.
    	\item Caso as informações sejam válidas o usuário deverá ser cadastrado no sistema.
    \end{itemize}
    \item US13: \textbf{Eu} como usuário, \textbf{desejo} editar meu perfil. \textbf{Para que eu possa} atualizar minhas informações
    \begin{itemize}
    	\item Ao entrar na aplicação, e obter êxito ao realizar login em sua conta, deve ser mostrado um quadro com as informações necessárias para gerenciar a conta.
    	\item O usuário então terá a possibilidade de alterar seu telefone, senha ou email. Os critérios de validação desses campos são realizados da seguinte forma: telefone é constituído somente por números, senhas deverão ter um tamanho mínimo de 6 caracteres e o email deverá ser na forma example@example.com.
    	\item Caso algum dado digitado não passe na validação uma mensagem deve ser informada ao usuário pelo o motivo do dado não ter sido aceito.
    	\item Caso as informações sejam válidas as informações do usuário modificadas deverão ser atualizadas no sistema.
    \end{itemize}
        \item US14: \textbf{Eu} como administrador, \textbf{desejo} ser capaz de deletar perfis de usuário. \textbf{Para que eu possa} remover contas de usuário que não trabalham mais na empresa
    \begin{itemize}
    	\item É necessário estar logado com uma conta de administrador. 
    	\item O administrador deverá conseguir deletar qualquer conta que possua um nível de permissão abaixo de administrador.
    	\item Para o administrador deve existir um menu que não aparece para os usuários.
    	\item No menu de administrador, é possível visualizar os usuários. Ao clicar no usuário deve aparecer as opções de gerência, como deletar.
    	\item Ao clicar em deletar deve aparecer uma janela de confirmação, garantido que um usuário seja deletado por acaso.
    	\item Caso um usuário seja deletado seus projetos continuam como estão.
    \end{itemize}
  \end{itemize}

